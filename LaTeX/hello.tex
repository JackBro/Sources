\documentclass{article}
\begin{document}
\title{How to Structure a \LaTex{} Document}
\author{qyh\\Ending}
\date{\today}
\maketitle
\begin{abstract}
  In this article
  I'm the abstract
  Its \LateX{} source,
  I'm the abstract again
\end{abstract}

\section{Introduction}
\label{sec:introduction}

This small is illustrate how easy it is to create a well structured document whthin \LateX\cite{lamport94}. You should .....

\section{Structure}
\label{sec:structure}

One of the greate advantages of \LaTex{} is ...
you tell \LaTeX{} what it needs to know ...

\section{Top Matter}
\label{sec:top-matter}

the first thing you normally have ...
In \LaTeX{} terms,
to as \emph{top matter}.

\subsection{Article Information}
\label{sec:article-information}

\begin{itemize}
\item \verb|\title{}| -- The title of the article
\item \verb|\date| -- The date. Use:
  \begin{itemize}
  \item \verb|\date{\today}| -- to get the date that the document is typeset.
  \item \verb|\date{}| -- for no date.
  \end{itemize}
  
\subsubsection{Author Information}
\label{sec:author-information}

The basic article class only provides the one command:
\begin{itemize}
\item \verb|\author{}| -- the author of the document.
\end{itemize}
It is common to not only include the author name, but to insert new line (\verb|\\|) after and add things such as address and email details. For a slightly more logical approach, use the AMS article class (\emph{amsart}) and you have the following extra commands:
\begin{itemize}
\item \verb|address| --- The author's address. Use the new line command (\verb|\\|) for line breaks.
\item \verb|thanks| --- where you put any acknowledgments.
\item \verb|email| --- The author's email address.
\item \verb|urladdr| --- the URL for the author's web page.
\end{itemize}

\subsection{Sectioning Commands}
\label{sec:sectioning-commands}

The commands for inserting sections are fairly intuitive. Of course,
certain commands are appropriate to different document classes.
For example, a book has chapters but a article doesn't.
\begin{center}
  \begin{tabular}{|l|l|}
    \hline
    Command & Level \\ \hline
    \verb|\part{}| & -1 \\
    \verb|chapter{}| & 0 \\
    \verb|section{}| & 1 \\
    \verb|subsection{}| & 2 \\
    \verb|subsubsection{}| & 3 \\
    \verb|paragraph{}| & 4 \\
    \verb|subparagraph{}| & 5 \\
    \hline
  \end{tabular}
\end{center}
\begin{thebibliography}{99}
\bibitem{lamport94}
  Lesile Lamport,
  \emph{\LaTeX: A Document Preparation System}.
  Addison Wesley, Massachusetts,
  2nd Edition,
  1994
\bibitem{wikibooks}
  http://en.wikibooks.org/wiki/LaTeX/simple.tex
\end{thebibliography}

\end{itemize}
\end{document}
